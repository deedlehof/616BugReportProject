\documentclass[]{article}

\usepackage{titlesec}
\usepackage[margin=1in]{geometry}

\newenvironment{logentry}[2]% date, heading
{\noindent\vspace{.2cm}\large\textbf{#2}\titlerule\large\textbf{#1}}
{\\\\\\\vspace{5cm}}

\begin{document}
\begin{logentry}{01/20/20}{{\LaTeX}, Google Format, and DE}
    \par The first week after officially starting my project has been busy.
    I started with trying to improve my development environment in the 
    hopes that it will make writing the project easier.  I'm using vim
    for all of my development.  Vim has a lot of nice plugins that I've 
    added to make my life easier.  For example, Ultisnips which allows
    for the creation and insertion of code snippets to streamline
    coding and prevent you from writing the same thing over and
    over again.
    \par Once I was mostly happy with my setup, I turned my attention
    to my project.  I've written a good chunk of code already for it
    from when I was in CS616, but it needed some work.  I spent some
    time cleaning up what was already there and making the project more
    concise.  For example, I was caching the important words from each of 
    the scanned source files in my home directory, so I moved it to the root
    of the project structure instead.  I also moved my git out a level so that
    I could include the testing directory - a future project, along with other
    fixes.
    \par The biggest fix to my code was formatting.  I had my own internalized 
    style when writing, so it was fairly consistent, but it wasn't perfect.
    While figuring out my DE I added a plugin for linting and got it working
    for the project's language - Java.  I went through and updated all of my
    code to follow Google's standard for Java development, minus their rule
    on tabs.  The indentation is correct, but I'm a fan of tabs over spaces 
    as all sane humans should be.  All that's left for that is writing the
    Javadoc comments.  Once that's complete it'll all be compliant.
    \par The last thing I worked on was learning {\LaTeX}. In all my
    time at university I've never sat down to learn {\LaTeX}, but with
    such a long term writing project - these logs - I decided
    to invest the time and I can already tell it was a good choice.
    {\LaTeX} is very powerful and I've got it automated with plugins 
    Ultisnips and vimtex, so that it is extremely easy to update when I make 
    changes to the project.
\end{logentry}
\end{document}
